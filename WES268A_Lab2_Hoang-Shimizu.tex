% !TeX program = pdflatex
% !BIB program = biber

\documentclass[11pt,a4paper]{article}

% PREAMBLE
\usepackage[margin=1in]{geometry}
\usepackage{graphicx,booktabs,amsmath,amssymb,mathtools,float}
\usepackage{siunitx}\sisetup{detect-all}
\usepackage[hidelinks]{hyperref}
\usepackage{titlesec}
\usepackage{tocloft}
\usepackage{newtxtext,newtxmath} % Times New Roman

% TOC Depth
\setcounter{secnumdepth}{4}
\setcounter{tocdepth}{4}

% Section Numbering
\renewcommand\thesection{\arabic{section}}
\renewcommand\thesubsection{\thesection.\arabic{subsection}}
\renewcommand\thesubsubsection{\thesubsection.\Alph{subsubsection}}

\makeatletter
\renewcommand\theparagraph{\thesubsubsection.\roman{paragraph}}
\makeatother

\titleformat{\paragraph}[hang]{\normalfont\normalsize\bfseries}{\theparagraph}{1em}{}
\titlespacing*{\paragraph}{0pt}{0.75ex}{0.5ex}

% TOC Indentation (skip paragraph due to MiKTeX compatibility)
\setlength{\cftsubsecindent}{1.5em}
\setlength{\cftsubsubsecindent}{3em}

% BEGIN DOCUMENT 
\begin{document}

% Title Page
\begin{titlepage}
    \centering
    {\Huge\bfseries UCSD MAS WES268A - Lab 2 Report DRAFT\par}
    \vspace{2cm}
    {\large\bfseries 08NOV2025\par}
    \vspace{2cm}
    \includegraphics[width=0.35\textwidth]{UCSD.png}\par\vspace{1.5cm}
    %\vfill
    \vspace{3cm}
    {\scshape Prepared By: \par}
    {\Large\itshape Joshua Hoang \\ Ryan Shimizu\par}

    % Good Boy Brady
    \vspace*{\fill}
    \hfill
    \includegraphics[width=0.2\textwidth]{EasterEgg.png}
\end{titlepage}

% Reset page number after title
\clearpage
\setcounter{page}{2}

% Table of Contents 
\tableofcontents
\newpage

% SECTION: Part 1
\section[Generation of Passband Noise Waveforms]
{Part 1: \textbf{Generation of Passband Noise Waveforms}}

% SUBSECTION: Narrative Questions
\subsection{Narrative Questions}

% 1.1.2
\hspace*{1.5em}\textbf{1.1.2 - } 
{The center value of the histograms \textit{I} and \textit{Q} is zero. 
This is because the noise is generated using a Gaussian 
distribution with a mean of zero.}

\textbf{1.1.3 - }
{From Figure 2, we can see that the \textit{IQ} spectrums show a flat 
frequency response indicating that the noise power is uniformly distributed 
across the frequency range hovering around the -40dB range. This is a 
characteristic of additive white Gaussian noise, which has equal power 
across all frequencies. Using an RBW of 50 Hz and one-sided bandwidth of 250kHz, 
we can calculate the total noise power as follows:
\begin{equation*}
    P_{\text{noise}} = (10^{-4}) \times B/RBW = (10^{-4}) \times (250,000/50) = 0.5
\end{equation*}
We can verify that this is correct because it is consistent with our value of $\sigma$:
\begin{equation*}
    \sigma_{\text{noise}} = \sqrt{P_{\text{noise}}} = \sqrt{0.5} = 0.7071
\end{equation*}
which matches our set value of $\sigma$ in the simulation VI seen in Figure 1 and Figure 2.
}

\textbf{1.1.4 - }
{Just like in step 3, we can calculate the total noise power using the same
 method but only for the in-phase component. Using an RBW of 50 Hz, 
 one-sided bandwidth of 250kHz, and a power level of -40dB uniformly distributed
 across the frequency range, we can calculate the total noise power as follows:
\begin{equation*}
    P_{I} = (10^{-4}) \times B/RBW = (10^{-4}) \times (250,000/50) = 0.5
\end{equation*}
Again, we can verify that this is correct because it is consistent with our value of $\sigma$:
\begin{equation*}
    \sigma_{I} = \sqrt{P_{I}} = \sqrt{0.5} = 0.7071
\end{equation*}
which matches. LabVIEW only displays the one-sided spectrum for real signals because the 
negative frequency components mirror the positive frequency components. Therefore, it is redundant to display.}

\subsection{Figures}
\begin{figure}[H]
   \centering
   \includegraphics[width=0.7\textwidth]{lab_ss/p1_2.png}
   \caption{1.1.2 - Histogram of \textit{I} and \textit{Q} components of passband noise}\label{fig:1}
\end{figure}
\begin{figure}[H]
   \centering
   \includegraphics[width=0.7\textwidth]{lab_ss/p1_3.png}
   \caption{1.1.3 - Spectrum of \textit{I} and \textit{Q} components of passband noise}\label{fig:2}
\end{figure}

\newpage

% SECTION: Part 2
\section[Up Sampling and Pulse Shaping]
{Part 2: \textbf{Up Sampling and Pulse Shaping}}

\subsection{Narrative Questions}

\hspace*{1.5em}\textbf{2.1.5 - }
{As we vary the Filter Parameter $\beta$ of the Root Raised Cosine Filter, we observe
 changes in the power spectrum of the shaped signal. Specifically, as $\beta$ increases,
 the bandwidth of the signal also increases. We can see this in Figure 2. This is because a higher $\beta$ value
 results in a wider transition band in the filter's frequency response. Conversely, 
 a lower $\beta$ value results in a narrower bandwidth.}

\subsection{Figures}
\begin{figure}[H]
   \centering
   \includegraphics[width=0.7\textwidth]{lab_ss/p2_3.png}
   \caption{2.1.3 - Side-by-side TX and RX VI}\label{fig:3}
\end{figure}
\begin{figure}[H]
   \centering
   \includegraphics[width=0.7\textwidth]{lab_ss/p2_4a.png}
   \caption{2.1.4a - Power Spectrum (No filter)}\label{fig:4}
\end{figure}
\begin{figure}[H]
   \centering
   \includegraphics[width=0.7\textwidth]{lab_ss/p2_4b1.png}
   \caption{2.1.4b1 - Power Spectrum of $\beta = 0$ (Root Raised Cosine Filter)}\label{fig:5}
\end{figure}
\begin{figure}[H]
   \centering
   \includegraphics[width=0.7\textwidth]{lab_ss/p2_4b2.png}
   \caption{2.1.4b2 - Power Spectrum of $\beta = 0.5$ (Root Raised Cosine Filter)}\label{fig:6}
\end{figure}
\begin{figure}[H]
   \centering
   \includegraphics[width=0.7\textwidth]{lab_ss/p2_4b3.png}
   \caption{2.1.4b3 - Power Spectrum of $\beta = 1$ (Root Raised Cosine Filter)}\label{fig:7}
\end{figure}

% SECTION: Part 3
\section[Generation of Pseudo Random Binary Sequences (PRBS)]
{Part 3: \textbf{Generation of Pseudo Random Binary Sequences (PRBS)}}

\subsection{Narrative Questions}

\hspace*{1.5em}\textbf{3.0.5 - }
{We analyze the power spectrum for this type of signal using a Fourier series expansion 
to represent the PRBS signal because it is periodic. Since the sequence is pseudo-random,
it has a period that repeats depending on the length of the shift registers used to generate it. For our case,
with $M = 5$ and $N = 8$, the period is $N \times (2^{M}-1) = 248$ samples.
This makes a Fourier series expansion appropriate for analyzing the frequency components of the signal.}

\textbf{3.0.6 - }
{
   The frequency spacing between the spectral lines $\Delta f$ is approximately 60kHz as seen in Figure 8. The
   $\Delta f_{\text{null}}$ repeats at 2kHz intervals as seen in Figure 10.
   We can verify that these are consistent with our calculations in Additional Questions 1:
   \begin{equation*}
   \Delta f = \frac{f_s}{N \times (2^{M}-1)} \qquad
   \Delta f_{\text{null}} = \frac{f_s}{N}
   \end{equation*}
   where $f_s = 500\text{kHz}$, $N = 8$ and $M = 5$.
   \begin{equation*}
   \Delta f = \frac{500\text{kHz}}{8 \times (2^{5}-1)} \approx 2.016\text{kHz} \qquad
   \Delta f_{\text{null}} = \frac{500\text{kHz}}{8} = 62.5\text{kHz}
   \end{equation*}
}

% SECTION: Part 3 Figures
\subsection{Figures}
\begin{figure}[H]
   \centering
   \includegraphics[width=0.7\textwidth]{lab_ss/p3_3.png}
   \caption{3.0.3 - Generated Spectrum using USRP Spectrum Analyzer}\label{fig:8}
\end{figure}
\begin{figure}[H]
   \centering
   \includegraphics[width=0.7\textwidth]{lab_ss/p3_4a.png}
   \caption{3.0.4a - DC Component of PRBS Spectrum}\label{fig:9}
\end{figure}
\begin{figure}[H]
   \centering
   \includegraphics[width=0.7\textwidth]{lab_ss/p3_6.png}
   \caption{3.0.6 - Frequency Spacing between spectral lines $\Delta f_{\text{null}}$ }\label{fig:10}
\end{figure}

\subsection{Autocorrelation Properties of PRBS Sequences (Simulation Only)}
\subsubsection{Narrative Questions}
\hspace*{1.5em}\textbf{3.1.2 -} 
{The results of the MLS, Ones, and Random sequences shown in Figures 11 through 13 
 match the expected behavior of the three sequences. When the noise standard deviation is set to 0,
 the MLS sequence achieves a detection rate very close to one because its autocorrelation always 
 gives a single sharp peak at $t = 0$. The Ones sequence has a flat correlation surface so the 
 algorithm rarely singles out $t = 0$, and its detection rate stays near zero. As for the Random 
 sequence, each newly generated pattern is essentially uncorrelated with the stored template, 
 so the peak almost never lands at $t = 0$ and the detection rate likewise remains close to zero.}

\textbf{3.1.4 -} 
{ Across all three noise levels the MLS sequence delivers the best performance as it consistently 
  shows the highest Packet Detection Rate in Figures 11, 14, 17, 20 and 23 compared to the Ones and Random 
  sequences in Figures 12, 13, 15, 16, 18, 19, 21 and 22.}

\textbf{3.1.5 -} 
{ Under additive white Gaussian noise the MLS sequence still performs the best. Figure 
  20 shows that the MLS maintains a clear correlation spike, while the Ones and Random sequences 
  in Figures 21 and 22 lose their peaks first.}

% SECTION: Part 3 Simulation Section Figures
\subsubsection{Figures}
% 3.1.2 Figures
\begin{figure}[H]
   \centering
   \includegraphics[width=0.7\textwidth]{lab_ss/p31_2_mls.png}
   \caption{3.1.2 - Correlation function for MLS}\label{fig:11}
\end{figure}
\begin{figure}[H]
   \centering
   \includegraphics[width=0.7\textwidth]{lab_ss/p31_2_ones.png}
   \caption{3.1.2 - Correlation function for Ones}\label{fig:12}
\end{figure}
\begin{figure}[H]
   \centering
   \includegraphics[width=0.7\textwidth]{lab_ss/p31_2_random.png}
   \caption{3.1.2 - Correlation function for Random}\label{fig:13}
\end{figure}

% 3.1.4a Figures
\begin{figure}[H]
   \centering
   \includegraphics[width=0.7\textwidth]{lab_ss/p31_4_a_mls.png}
   \caption{3.1.4a - Correlation function for MLS with $\beta = 0.5$}\label{fig:14}
\end{figure}
\begin{figure}[H]
   \centering
   \includegraphics[width=0.7\textwidth]{lab_ss/p31_4_a_ones.png}
   \caption{3.1.4a - Correlation function for Ones with $\beta = 0.5$}\label{fig:15}
\end{figure}
\begin{figure}[H]
   \centering
   \includegraphics[width=0.7\textwidth]{lab_ss/p31_4_a_random.png}
   \caption{3.1.4a - Correlation function for Random with $\beta = 0.5$}\label{fig:16}
\end{figure}

% 3.1.4b Figures
\begin{figure}[H]
   \centering
   \includegraphics[width=0.7\textwidth]{lab_ss/p31_4_b_mls.png}
   \caption{3.1.4b - Correlation function for MLS with $\beta = 1.0$}\label{fig:17}
\end{figure}
\begin{figure}[H]
   \centering
   \includegraphics[width=0.7\textwidth]{lab_ss/p31_4_b_ones.png}
   \caption{3.1.4b - Correlation function for Ones with $\beta = 1.0$}\label{fig:18}
\end{figure}
\begin{figure}[H]
   \centering
   \includegraphics[width=0.7\textwidth]{lab_ss/p31_4_b_random.png}
   \caption{3.1.4b - Correlation function for Random with $\beta = 1.0$}\label{fig:19}
\end{figure}

% 3.1.4c Figures
\begin{figure}[H]
   \centering
   \includegraphics[width=0.7\textwidth]{lab_ss/p31_4_c_mls.png}
   \caption{3.1.4c - Correlation function for MLS with $\beta = 2.0$}\label{fig:20}
\end{figure}
\begin{figure}[H]
   \centering
   \includegraphics[width=0.7\textwidth]{lab_ss/p31_4_c_ones.png}
   \caption{3.1.4c - Correlation function for Ones with $\beta = 2.0$}\label{fig:21}
\end{figure}
\begin{figure}[H]
   \centering
   \includegraphics[width=0.7\textwidth]{lab_ss/p31_4_c_random.png}
   \caption{3.1.4c - Correlation function for Random with $\beta = 2.0$}\label{fig:22}
\end{figure}

% 3.1.5 Figures
\begin{figure}[H]
   \centering
   \includegraphics[width=0.7\textwidth]{lab_ss/p31_5_mls_best.png}
   \caption{3.1.5 - Correlation of Best Performing Sequence with $\beta = 2.0$ (MLS)}\label{fig:23}
\end{figure}

% SECTION: Part 3 Additional Questions
\subsection{Additional Questions}
\hspace*{1.5em}\textbf{1 - }
\begin{equation*}
T_{\text{sym}} = \frac{N}{f_s}, \qquad T_{\text{period}} = \frac{N \times(2^{M}-1)}{f_s}
\end{equation*}
\begin{equation*}
\Delta f = \frac{1}{T_{\text{period}}} = \frac{f_s}{N,(2^{M}-1)}, \qquad
\Delta f_{\text{null}} = \frac{1}{T_{\text{sym}}} = \frac{f_s}{N}
\end{equation*}

\textbf{2 - }
{
   PRBS sequences are used for frame synchronization because it uses a known, repeatable bit pattern 
   whose autocorrelation has one strong, well-defined peak at zero lag. In practical terms this 
   gives a clear correlation spike even at relatively low SNR, so the receiver can find the 
   start of the frame quickly and reliably while still using a preamble with a fairly flat spectrum 
   that stays within emission limits. When a receiver correlates the incoming signal with a locally 
   generated PRBS sequence, the correlation output will produce a distinct peak when the sequences 
   align indicating that a message is currently being sent. These sequences also have very little 
   cross correlation with other sequences, making them useful for distinguishing between different 
   users on the same channel.
}

% SECTION: Part 4
\section[Eye Patterns (Simulation VI)]
{Part 4: \textbf{Eye Patterns (Simulation VI)}}

\subsection{Narrative Questions}

\hspace*{1.5em}\textbf{4.0.4 -} 
{
   Figure 19 shows that the largest value of Zeta occurs when the receiver samples at the centre of 
   the eight-sample symbol, which is the fourth sample position. At that instant Zeta is 9.80653 for 
   a raised-cosine pulse with Beta equal to 1.
 }

\textbf{4.0.5a -} 
{
   For Beta equal to zero, Zeta plunges from about ten to roughly two within one sample on either side
   of the optimum in Figures 20 through 22. With Beta equal to one-half the fall-off is even sharper, 
   dropping to almost zero in Figures 23 through 25. By contrast, the broader pulse in Figure 19 
   with Beta equal to one loses only a fraction of a decibel over the same shift, showing that wider 
   roll-off is far less sensitive to sampling error.
}

\textbf{4.0.5b -} 
{
   Beta equal to one-half is the most sensitive to sampling time because Zeta collapses from about 
   10.35 to essentially zero with a single-sample shift.
}

\textbf{4.0.5c -} 
{ 
   Beta equal to one is the least sensitive. The wider roll-off spreads the pulse energy so 
   inter-symbol interference grows more slowly when the sampling instant moves away from the optimum, 
   leaving Zeta almost unchanged across neighbouring samples, as suggested by the gentle eye opening 
   in Figure 19.
}

\textbf{4.0.6a -} 
{
   With no transmit filter the rectangular pulse in Figure 26 gives a peak Zeta of about 9.97. When 
   moving the sampling point a single sample early or late, the eye closes noticeably faster than it 
   does for the raised-cosine pulse with Beta equal to one shown in Figure 19, yet not nearly as 
   abruptly as for the tighter roll-off cases in Figures 20 through 25. As a result, its timing 
   sensitivity resides between those two extremes.
}

\textbf{4.0.6b -} 
{
   The peak Zeta of 9.97 for the unfiltered case is only a couple of percent higher than the 9.81 
   recorded for Beta equal to one in Figure 19, so the two optimal values are practically the same. 
   The key difference is not the height of the peak, but how rapidly Zeta drops when the timing 
   slides away from it.
}

\textbf{4.0.7 -} 
{
   At the best sampling instant Zeta is about 10.30 for Beta equal to zero, 10.35 for Beta equal 
   to one half, 9.81 for Beta equal to one, and 9.97 for the unfiltered rectangular pulse. Narrower 
   roll-off factors concentrate more signal energy at the center of the symbol, giving a slightly 
   larger eye opening, but they also tighten the transition slopes so any timing error hurts much 
   more. A wider roll-off sacrifices a small amount of peak Zeta yet offers a more forgiving eye 
   diagram.
}

% SECTION: Part 4 Figures
\subsection{Figures}
\begin{figure}[H]
    \centering
    \includegraphics[width=0.7\textwidth]{lab_ss/p4_3_threehist.png}
    \caption{[4.0.3: Nth Sample 4 - Three Histograms]}
    \label{fig:4.0.3threehist}
\end{figure}

\begin{figure}[H]
    \centering
    \includegraphics[width=0.7\textwidth]{lab_ss/p4_3_fourhist.png}
    \caption{[4.0.3: Nth Sample 5 - Four Histograms]}
    \label{fig:4.0.3fourhist}
\end{figure}

\begin{figure}[H]
    \centering
    \includegraphics[width=0.7\textwidth]{lab_ss/p4_4_rc.png}
    \caption{[4.0.4: Eye Trace and Histogram at Optimal Time [Beta = 1 | Zeta = 9.80653]]}
    \label{fig:4.0.4}
\end{figure}

\begin{figure}[H]
    \centering
    \includegraphics[width=0.7\textwidth]{lab_ss/p4_5_beta00_twohist.png}
    \caption{[4.0.5: Two Histograms [Beta = 0 | Zeta = 10.301]]}
    \label{fig:4.0.5}
\end{figure}

\begin{figure}[H]
    \centering
    \includegraphics[width=0.7\textwidth]{lab_ss/p4_5_beta00_threehist.png}
    \caption{[4.0.5: Three Histograms [Beta = 0 | Zeta = 0.331194]]}
    \label{fig:4.0.5}
\end{figure}

\begin{figure}[H]
    \centering
    \includegraphics[width=0.7\textwidth]{lab_ss/p4_5_beta00_fourhist.png}
    \caption{[4.0.5: Four Histograms [Beta = 0 | Zeta = 2.12025]]}
    \label{fig:4.0.5}
\end{figure}

\begin{figure}[H]
    \centering
    \includegraphics[width=0.7\textwidth]{lab_ss/p4_5_beta05_twohist.png}
    \caption{[4.0.5: Two Histograms [Beta = 0.5 | Zeta = 10.345]]}
    \label{fig:4.0.5}
\end{figure}

\begin{figure}[H]
    \centering
    \includegraphics[width=0.7\textwidth]{lab_ss/p4_5_beta05_threehist.png}
    \caption{[4.0.5: Three Histograms [Beta = 0.5 | Zeta = 0.015093]]}
    \label{fig:4.0.5}
\end{figure}

\begin{figure}[H]
    \centering
    \includegraphics[width=0.7\textwidth]{lab_ss/p4_5_beta05_fourhist.png}
    \caption{[4.0.5: Four Histograms [Beta = 0.5 | Zeta = 1.17704]]}
    \label{fig:4.0.5}
\end{figure}

\begin{figure}[H]
    \centering
    \includegraphics[width=0.7\textwidth]{lab_ss/p4_6.png}
    \caption{[4.0.6: Transmit Filter None [Zeta = 9.96767]]}
    \label{fig:4.0.6}
\end{figure}

% SECTION: Part 4.1
\subsection{Interference and Low-Pass Filtering}
\subsubsection{Narrative Questions}

\hspace*{1.5em}\textbf{[4.1.1a]} 
{
   When the Butterworth cut-off sits at two-hundred percent of the symbol rate 
   the eye remains wide and clean in both the raised-cosine capture (Figure 27) 
   and the unfiltered capture (Figure 30). Dropping the cut-off to one-hundred 
   percent trims high-frequency content so the openings shrink noticeably, as 
   seen in Figures 28 and 31. A further drop to fifty percent removes much of 
   the spectrum that separates adjacent symbols; inter-symbol interference 
   smears the crossings and the eyes in Figures 29 and 32 are almost closed. 
   The progressive closure happens because a tighter bandwidth stretches each 
   symbol in time, making successive bits overlap and crowd the decision point.
}

\textbf{[4.1.1a.i]} 
{
   The low-pass filter makes the signal more vulnerable to inter-symbol interference 
   than the raised-cosine pulse-shaping alone. The raised-cosine waveform is already 
   designed to satisfy the Nyquist criterion, so even after the extra filtering 
   its eye retains a clearer opening, whereas the unfiltered sequence loses 
   definition much faster at the same cut-off settings.
}
\subsubsection{Figures}
\begin{figure}[H]
    \centering
    \includegraphics[width=0.7\textwidth]{lab_ss/p41_rcTX_200percent.png}
    \caption{[Cut-Off Frequency 200 Percent - Raised Cosine Filter]}
    \label{fig:4.1}
\end{figure}

\begin{figure}[H]
    \centering
    \includegraphics[width=0.7\textwidth]{lab_ss/p41_rcTX_100percent.png}
    \caption{[Cut-Off Frequency 100 Percent - Raised Cosine Filter]}
    \label{fig:4.1}
\end{figure}

\begin{figure}[H]
    \centering
    \includegraphics[width=0.7\textwidth]{lab_ss/p41_rcTX_50percent.png}
    \caption{[Cut-Off Frequency 50 Percent - Raised Cosine Filter]}
    \label{fig:4.1}
\end{figure}

\begin{figure}[H]
    \centering
    \includegraphics[width=0.7\textwidth]{lab_ss/p41_nofilterTX_200percent.png}
    \caption{[Cut-Off Frequency 200 Percent - Un-Filtered]}
    \label{fig:4.1}
\end{figure}

\begin{figure}[H]
    \centering
    \includegraphics[width=0.7\textwidth]{lab_ss/p41_nofilterTX_100percent.png}
    \caption{[Cut-Off Frequency 100 Percent - Un-Filtered]}
    \label{fig:4.1}
\end{figure}

\begin{figure}[H]
    \centering
    \includegraphics[width=0.7\textwidth]{lab_ss/p41_nofilterTX_50percent.png}
    \caption{[Cut-Off Frequency 50 Percent - Un-Filtered]}
    \label{fig:4.1}
\end{figure}

% Easter Egg
\newpage
\thispagestyle{empty}
\vspace*{\fill}
\begin{center}
   \includegraphics[width=0.4\textwidth]{EasterEgg2.png}\\[1em]
   \textbf{feelsgoodman}
\end{center}
\vspace*{\fill}

\end{document}
