% !TeX program = pdflatex
% !BIB program = biber

\documentclass[11pt,a4paper]{article}

% PREAMBLE
\usepackage[margin=1in]{geometry}
\usepackage{graphicx,booktabs,amsmath,amssymb,mathtools,float}
\usepackage{siunitx}\sisetup{detect-all}
\usepackage[hidelinks]{hyperref}
\usepackage{titlesec}
\usepackage{tocloft}
\usepackage{newtxtext,newtxmath} % Times New Roman

% TOC Depth
\setcounter{secnumdepth}{4}
\setcounter{tocdepth}{4}

% Section Numbering
\renewcommand\thesection{\arabic{section}}
\renewcommand\thesubsection{\thesection.\arabic{subsection}}
\renewcommand\thesubsubsection{\thesubsection.\Alph{subsubsection}}

\makeatletter
\renewcommand\theparagraph{\thesubsubsection.\roman{paragraph}}
\makeatother

\titleformat{\paragraph}[hang]{\normalfont\normalsize\bfseries}{\theparagraph}{1em}{}
\titlespacing*{\paragraph}{0pt}{0.75ex}{0.5ex}

% TOC Indentation (skip paragraph due to MiKTeX compatibility)
\setlength{\cftsubsecindent}{1.5em}
\setlength{\cftsubsubsecindent}{3em}

% BEGIN DOCUMENT 
\begin{document}

% Title Page
\begin{titlepage}
    \centering
    {\Huge\bfseries UCSD MAS WES268A - Lab 2 Report DRAFT\par}
    \vspace{2cm}
    {\large\bfseries 08NOV2025\par}
    \vspace{2cm}
    \includegraphics[width=0.35\textwidth]{UCSD.png}\par\vspace{1.5cm}
    %\vfill
    \vspace{3cm}
    {\scshape Prepared By: \par}
    {\Large\itshape Joshua Hoang \\ Ryan Shimizu\par}

    % Good Boy Brady
    \vspace*{\fill}
    \hfill
    \includegraphics[width=0.2\textwidth]{EasterEgg.png}
\end{titlepage}

% Reset page number after title
\clearpage
\setcounter{page}{2}

% Table of Contents 
\tableofcontents
\newpage

% SECTION: Part 1
\section[Generation of Passband Noise Waveforms]
{Part 1: \textbf{Generation of Passband Noise Waveforms}}

% SUBSECTION: Narrative Questions
\subsection{Narrative Questions}

% 1.1.2
\hspace*{1.5em}\textbf{1.1.2 - } 
{The center value of the histograms \textit{I} and \textit{Q} is zero. 
This is because the noise is generated using a Gaussian 
distribution with a mean of zero.}

\textbf{1.1.3 - }
{From Figure 2, we can see that the \textit{IQ} spectrums show a flat 
frequency response indicating that the noise power is uniformly distributed 
across the frequency range hovering around the -40dB range. This is 
characteristic of additive white Gaussian noise, which has equal power 
across all frequencies. Using an RBW of 50 Hz and one-sided bandwidth of 250kHz, 
we can calculate the total noise power as follows:
\begin{equation*}
    P_{noise} = (10^{-4}) \times B/RBW = (10^{-4}) \times (250,000/50) = 0.5
\end{equation*}
We can verify that this is correct because it is consistent with our value of $\sigma$:
\begin{equation*}
    \sigma_{noise} = \sqrt{P_{noise}} = \sqrt{0.5} = 0.7071
\end{equation*}
which matches our set value of $\sigma$ in the simulation VI seen in Figure 1 and Figure 2.
}

\textbf{1.1.4 - }
{Just like in step 3, we can calculate the total noise power using the same
 method but only for the in-phase component. Using an RBW of 50 Hz, 
 one-sided bandwidth of 250kHz, and a power level of -40dB uniformly distributed
 across the frequency range, we can calculate the total noise power as follows:
\begin{equation*}
    P_{I} = (10^{-4}) \times B/RBW = (10^{-4}) \times (250,000/50) = 0.5
\end{equation*}
Again, we can verify that this is correct because it is consistent with our value of $\sigma$:
\begin{equation*}
    \sigma_{I} = \sqrt{P_{I}} = \sqrt{0.5} = 0.7071
\end{equation*}
which matches. LabVIEW only displays the one-sided spectrum for real signals because the 
negative frequency components mirror the positive frequency components. Therefore, it is redundant to display.}

\subsection{Figures}
\begin{figure}[H]
   \centering
   \includegraphics[width=0.7\textwidth]{lab_ss/p1_2.png}
   \caption{1.1.2 - Histogram of \textit{I} and \textit{Q} components of passband noise}\label{fig:1}
\end{figure}
\begin{figure}[H]
   \centering
   \includegraphics[width=0.7\textwidth]{lab_ss/p1_3.png}
   \caption{1.1.3 - Spectrum of \textit{I} and \textit{Q} components of passband noise}\label{fig:2}
\end{figure}

\newpage

% SECTION: Part 2
\section[Up Sampling and Pulse Shaping]
{Part 2: \textbf{Up Sampling and Pulse Shaping}}

\subsection{Narrative Questions}

\hspace*{1.5em}\textbf{2.1.5 - }
{As we vary the Filter Parameter $\beta$ of the Root Raised Cosine Filter, we observe
 changes in the power spectrum of the shaped signal. Specifically, as $\beta$ increases,
 the bandwidth of the signal also increases. We can see this in Figure 2. This is because a higher $\beta$ value
 results in a wider transition band in the filter's frequency response. Conversely, 
 a lower $\beta$ value results in a narrower bandwidth.}

\textbf{[INSERT]} 

\subsection{Figures}
\begin{figure}[H]
   \centering
   \includegraphics[width=0.7\textwidth]{lab_ss/p2_3.png}
   \caption{2.1.3 - Side-by-side TX and RX VI}\label{fig:3}
\end{figure}
\begin{figure}[H]
   \centering
   \includegraphics[width=0.7\textwidth]{lab_ss/p2_4a.png}
   \caption{2.1.4a - Power Spectrum (No filter)}\label{fig:4}
\end{figure}
\begin{figure}[H]
   \centering
   \includegraphics[width=0.7\textwidth]{lab_ss/p2_4b1.png}
   \caption{2.1.4b1 - Power Spectrum of $\beta = 0$ (Root Raised Cosine Filter)}\label{fig:5}
\end{figure}
\begin{figure}[H]
   \centering
   \includegraphics[width=0.7\textwidth]{lab_ss/p2_4b2.png}
   \caption{2.1.4b2 - Power Spectrum of $\beta = 0.5$ (Root Raised Cosine Filter)}\label{fig:6}
\end{figure}
\begin{figure}[H]
   \centering
   \includegraphics[width=0.7\textwidth]{lab_ss/p2_4b3.png}
   \caption{2.1.4b3 - Power Spectrum of $\beta = 1$ (Root Raised Cosine Filter)}\label{fig:7}
\end{figure}

% SECTION: Part 3
\section[Generation of Pseudo Random Binary Sequences (PRBS)]
{Part 3: \textbf{Generation of Pseudo Random Binary Sequences (PRBS)}}

\subsection{Narrative Questions}

\hspace*{1.5em}\textbf{[INSERT]} 

\textbf{[INSERT]} 

%\subsection{[INSERT FIGURES]}
%\begin{figure}[H]
%    \centering
%    \includegraphics[width=0.7\textwidth]{FILENAME.png}
%    \caption{[INSERT]}
%    \label{fig:INSERT}
%\end{figure}

% SECTION: Part 4
\section[Eye Patterns (Simulation VI)]
{Part 4: \textbf{Eye Patterns (Simulation VI)}}

\subsection{Narrative Questions}

\hspace*{1.5em}\textbf{[INSERT]} 

\textbf{[INSERT]} 

%\subsection{[INSERT FIGURES]}
%\begin{figure}[H]
%    \centering
%    \includegraphics[width=0.7\textwidth]{FILENAME.png}
%    \caption{[INSERT]}
%    \label{fig:INSERT}
%\end{figure}

% Easter Egg
%\newpage
%\thispagestyle{empty}
%\vspace*{\fill}
%\begin{center}
%    \includegraphics[width=0.4\textwidth]{EasterEgg2.png}\\[1em]
%    \textbf{feelsgoodman}
%\end{center}
%\vspace*{\fill}

\end{document}
